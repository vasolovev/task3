\question
Ребята приехали в математический лагерь, где каждый получил футболку с уникальным номером от 1 до 23. Отправившись на очередной полдник, они обнаружили, что нет ни одного кекса – их украли! Ребята сразу приступили к расследованию. Таким образом, они сделали вывод, что виновник – не один человек, а целая группа! У них получилось разделить всех ребят на 4 группы подозреваемых, в зависимости от того, кто где был в предположительное время совершения преступления по словам очевидцев.
\\(Легенда: С – столовая, D – двор, B – баскетбольная площадка, А - аллея): 
\begin{center}
$A: \{1, 2, 3, 21, 23, 5, 22, 18, 19, 6\}$
\\
$B: \{6, 22, 10, 15, 11, 13, 7, 18, 14, 9\}$
\\
$C: \{7, 8, 14, 20, 12, 4, 1, 2, 19, 6\}$
\\
$D: \{9, 13, 16, 17, 18, 19, 22, 14, 5, 6\}$
\end{center}
Так как ребята были отличными математиками, у них получилось составить выражение, которое раскроет, кто виноват в преступлении. 
\begin{equation*}
    A \cup B \cap \overline{C} \cup (A \cap D \cup \overline{C}) \cup D
\end{equation*}
Помогите им найти виновных.

---------------

Автор -- Баженова Мария, М3219