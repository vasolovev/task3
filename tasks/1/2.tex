\question
Пусть у нас есть $P$ -- множество студентов города Санкт-Петербург, из них $A$ -- третьекурсники, $B$ -- проходят профессиональную переподготовку, а $C$ -- стажируются в Яндексе. Для статьи “Как учиться на третьем курсе университета ИТМО, проходить профессиональную подготовку и не умереть” Мегабайт создал выборку $D$.

Студенты ИТМО участвуют в специальной лотерее. Мы спросили номера лотерейных билетов некоторых из них. Получилась такая статистика:

\begin{center}
$A = \{1, 3, 7, 12, 15, 19, 22\}$
\\
$B = \{2, 3, 7, 9, 13, 16, 18, 21, 24\}$
\\
$C = \{2, 4, 5, 8, 10, 11, 13, 14, 17, 21, 23\}$
\\
$D = \{1, 2, 9, 13, 16, 19, 21, 22, 24\}$
\end{center}

Помогите Мегабайту понять, какие номера билетов у студентов из их выборки, которые стажируются в Яндексе.

---------------

Автор -- Антонина Чернова, М33081