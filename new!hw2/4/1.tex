\question
Идёт практика в ВУЗе по предмету Математический Анализ. На практике присутствует всего 5 учеников из группы 20 человек. Умник A, его друг B, а также ученики C, D, E. Учеников в группе так мало, потому что все остальные отчислены. Причиной такого массового отчисления стала следующая закономерность, выявленная на этих практиках за последнее время:

\begin{enumerate}
    \item Если A может решить задачу, то и B может её решить.
\item Хотя бы один из D или E могут решить задачу.
\item Либо B, либо C решали такую задачу раннее и могут решить её снова.
\item C или D либо совместно ботают и могут решить задачу, либо не решают её.
\item Если E решает задачу, то A и D понимают его логику и могут решить эту задачу.
\end{enumerate}
Кто из них может решить задачу на текущей паре, а кто нет?

---------------

Автор -- Вероника Вдовина, М3203