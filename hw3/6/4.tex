\question 
Постройте дерево по следующему коду: 1110101100100011001111001000 (1 - рисуем ребро, 0 - возвращаемся по ребру)

а) Посчитайте для полученного дерева (по формулам для дерева)
\begin{enumerate}
\item радиус;
\item диаметр;
\item центр (отметь на дереве);
\item число листьев (отметь на дереве);
\end{enumerate}

б) Дополните дерево 6 (шестью) ребрами так, чтобы число листьев и центров не изменилось:
\begin{enumerate}
\item нарисуйте новое дерево отдельно (покажите, что условия выполнены);
\item посчитайте для нового дерева: радиус, диаметр и центр (отметь на дереве);
\item отметьте все листья на  новом дереве;
\item закодируйте новое дерево;
\end{enumerate}

в) На основе полученных деревьев  постройте следующие графы:
\begin{enumerate}
\item граф, который имеет 7 циклов разной длины
\item граф, который имеет 11 компонент вершинной двусвязности;
\item граф, который имеет 13 компонент реберной двусвязности;
\item граф, у котрого цикломатическое число 9;
\end{enumerate}