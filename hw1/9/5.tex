\question
Два контрабандиста затеяли сделку. Им кажется, что все должно пройти гладко и их план идеален, но стражи порядка уже взялись за это дело и планируют встать между ними, сорвав аферу. $\{1, 2, 3, 0\}$ – товары, которые планирует передать контрабандист $A$. $\{1, 2, 3, 4\}$ – планирует передать $B$. $A$, ничего не подозревая, уже готов передать две штуки товара 1 в руки полицейскому (притворившегося контрабандистом $B$) в обмен на 1 и 3 товар подставного $B$, две штуки товарa 2 – тоже в обмен на 1 и 3 товар. Также полиция связалась с $B$ по поводу передачи товара 3 в обмен на 4, и 3 на 1. Таким образом, если у полиции получилось перехватить сделки с обоих концов – работа выполена, и товар не окажется в руках ни у $A$ ни у $B$. Какие и сколько сделок полиции удалось предотвратить?
\\
---------------

Автор -- Мария Баженова, М3219